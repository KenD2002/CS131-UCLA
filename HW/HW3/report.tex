\documentclass[letterpaper,twocolumn,10pt]{article}
\usepackage{usenix2019_v3}
\usepackage[top=1.2in, left=1.2in, bottom=1.2in, right=1.2in]{geometry}
\usepackage{booktabs}
\usepackage{array}

% to be able to draw some self-contained figs
\usepackage{tikz}
\usepackage{amsmath}

% inlined bib file
\usepackage{filecontents}

%-------------------------------------------------------------------------------
\begin{filecontents}{\jobname.bib}
%-------------------------------------------------------------------------------
@Book{arpachiDusseau18:osbook,
  author =       {Arpaci-Dusseau, Remzi H. and Arpaci-Dusseau Andrea C.},
  title =        {Operating Systems: Three Easy Pieces},
  publisher =    {Arpaci-Dusseau Books, LLC},
  year =         2015,
  edition =      {1.00},
  note =         {\url{http://pages.cs.wisc.edu/~remzi/OSTEP/}}
}
@InProceedings{waldspurger02,
  author =       {Waldspurger, Carl A.},
  title =        {Memory resource management in {VMware ESX} server},
  booktitle =    {USENIX Symposium on Operating System Design and
                  Implementation (OSDI)},
  year =         2002,
  pages =        {181--194},
  note =         {\url{https://www.usenix.org/legacy/event/osdi02/tech/waldspurger/waldspurger.pdf}}}
\end{filecontents}

%-------------------------------------------------------------------------------
\begin{document}
%-------------------------------------------------------------------------------

%don't want date printed
\date{}

% make title bold and 14 pt font (Latex default is non-bold, 16 pt)
\title{\Large \bf CS 131 Homework 3. Java shared memory performance races}

%for single author (just remove % characters)
\author{
{Ken Deng}\\
UCLA
% copy the following lines to add more authors
% \and
% {\rm Name}\\
%Name Institution
} % end author

\maketitle

%-------------------------------------------------------------------------------
\begin{abstract}
%-------------------------------------------------------------------------------
This project is trying to investigate the influence of different
synchronization and multi-thread techniques on performances
in Java programs. A main task is to implement synchronization
without ``synchronized" keyword, trading reliability (race 
condition) for better performance.
\end{abstract}
%-------------------------------------------------------------------------------
\section{Introduction}
%-------------------------------------------------------------------------------

We are evaluating the performances of three kinds of state 
management strategies (NullState, SynchronizedState, and
UnsynchronizedState) with different thread settings. The 
settings for our evaluation includes: number of threads,
state array entries, and swap transitions.\\

\noindent Specifically, we are doing the following tasks:\\
1. Creating an array of longs consisting of zeros;\\
2. Randomly select two array entries, and subtract one from
the first entry and minus one from the second entry.\\
3. Measures the CPU time for the overall operations.

\noindent The shell command we use for testing is similar to:
\begin{verbatim}
time timeout 3600 java UnsafeMemory Platform
5 Synchronized 8 100000000
\end{verbatim}
Platform says to use platform threads; 5 says to use a state
array of 5 entries; Synchronized means to test the 
SynchronizedState implementation; 8 means to divide the 
work into 8 threads of roughly equal size; 100000000 means
to do 100 million swap transitions total. Also note that if the
sum is not zero, then there are some reliability issues, e.g.
race condition.

%-------------------------------------------------------------------------------
\section{NullState}
%-------------------------------------------------------------------------------

The swap function in NullState is inactive and performs no
operations. This state demonstrates the quickest
synchronization possible, as it lacks any synchronization
mechanisms but still consistently returns a sum of zero. This
occurs because the swap function is non-functional and the
state array starts off initialized to zero.

\noindent\textbf{2.1 Number of threads}: as it increases, the 
total/average real time and user time both slightly increases.\\
\noindent\textbf{2.2 Size of Array}: no significant changes.\\
\noindent\textbf{2.3 Virtual/Platform}: no significant changes.

%-------------------------------------------------------------------------------
\section{SynchronizedState}
%-------------------------------------------------------------------------------

The swap function in SynchronizedState uses the ``synchronized"
keyword to ensure that the swap operation is accessed by only
one thread at a time. This means that while one thread is
performing a swap, no other thread can execute the swap
function simultaneously.

\noindent\textbf{3.1 Number of threads}: as it increases along 1,8, 40,
the total real time and user time both slightly increases. The sys
time remains roughly the same. Yet the average real time
is significantly increased.\\
\textbf{Explanation}: The decline in performance as the number
of threads increases can be attributed in part to possible lock
contention. This occurs when several threads try to run the
same section of code simultaneously, but only one thread
can advance at a time. The other threads are forced to wait
until the lock is released by the thread currently in operation.\\
\textbf{Reliability}: The uses of the ``synchronized" keyword
provides tools like thread locks, which prevents the race
condition between different threads. Thus the reliability is
enhanced.

\noindent\textbf{3.2 Size of Array}: as it increases from 5 to 100,
there is a slight decreasing tendency in the timings.\\
\textbf{Explanation}: the reason might be that as the array size
is increased, there is less probability that two threads would
compete for the same array element.

\noindent\textbf{3.3 Virtual/Platform}: virtual seems to have
longer timings than platform, especially when the number of
threads is larger than 1.\\
\textbf{Explanation}: the reason might be that the virtual thread
is a lightweight abstraction of a task that can be bound to a 
platform thread and is scheduled by the Java virtual thread 
scheduler, which requires mounting and therefore has more workload.

%-------------------------------------------------------------------------------
\section{UnsynchronizedState}
%-------------------------------------------------------------------------------

The swap function in UnsynchronizedState does not employ
the ``synchronized" keyword to ensure that the swap operation
is performed by only one thread at a time. Consequently, while
one thread is performing a swap, another thread may also
initiate a swap operation. This scenario illustrates the
significance of synchronization and the consequences of not
managing race conditions.

\noindent\textbf{4.1 Number of threads}: generally the timings
for it are less than SynchronizedState. As we increase along
1,8, 40, the user time and total real time increase slightly, but
average real time increases significantly.\\
\textbf{Explanation}: In the Unsynchronized State, the lack of
synchronization overhead results in quicker overall execution
times. However, the overhead associated with managing
threads remains. Even when compared to itself, an increase
in the number of threads will still lead to a rise in user time,
real time, and average swap time.\\
\textbf{Reliability}: The lack of the ``synchronized" keyword
leads to race condition, which yields in the sum not being zero.

\noindent\textbf{4.2 Size of Array}: as it increases from 5 to 100,
all the timings are increasing.\\
\textbf{Explanation}: as we are not using ``synchronized" keyword,
there is no more benefits of lowering the probability of two threads
accessing the same array entry. Moreover, as the size of array
increases, it has a more workload due to caching problem.

\noindent\textbf{4.3 Virtual/Platform}: all timings seem to be the same.\\
\textbf{Explanation}: as there is no synchronization, no lock would
happen, and thus there are no extra delay for waiting between threads.

%-------------------------------------------------------------------------------
\section{Conclusion}
%-------------------------------------------------------------------------------

\noindent The UnsynchronizedState is more effective in single-threaded
environments due to its higher efficiency. However, as the
number of threads grows, the SynchronizedState becomes
more beneficial due to its reliability, yet if there are too many
threads, delays can occur from managing locks.\\

\noindent Platform threads are particularly effective with the
SynchronizedState, while the impact on UnsynchronizedState
or NullState are not significant. On the other hand, the virtual
threads tend to increase system time.\\

\noindent Regarding the size of the state array, it represents
a balance between cache efficiency and the level of lock
contention. Therefore, the performance of the
SynchronizedState remains stable regardless of array size,
whereas the performance of the UnsynchronizedState
decreases significantly as the array size grows.\\

\noindent Benchmark for SynchronizedState: array size = 5 or 100;
number of threads = 8; Platform\\

\noindent Benchmark for UnsynchronizedState: array size = 5;
number of threads = 1; Platform or Virtual\\

%-------------------------------------------------------------------------------
\section{Difficulties}
%-------------------------------------------------------------------------------

The testing results are not stable, especially when the number
of threads is huge. This may because when running on the
lnxsrv11 server, there can be many other students running other
programs, resulting in workload and performance fluctuations.
Also, after using ``javac *.java" command to get the .class files,
these files are not compiled into static machine codes, but it will
be further recompiled in runtime, which causes unpredictability.\\

\noindent My solution involves running my program in late nights,
where there are less request from other students to the server.

\pagebreak

%-------------------------------------------------------------------------------
\section{Appendix}
%-------------------------------------------------------------------------------

\textbf{7.1 NullState}\\

\noindent Platform, size of state array = 5:\\
Threads = 1: Total real time 1.15471 s Average real swap time 11.5471 ns real  0m1.291s user  0m1.280s sys     0m0.040s\\
Threads = 8: Total real time 0.416577 s Average real swap time 33.3261 ns real  0m0.551s user  0m1.669s sys   0m0.044s\\
Threads = 40: Total real time 0.431523 s
Average real swap time 172.609 ns
real    0m0.569s
user    0m1.710s
sys     0m0.056s\\

\noindent Platform, size of state array = 100:\\
Threads = 1: Total real time 1.43610 s
Average real swap time 14.3610 ns
real    0m1.583s
user    0m1.568s
sys     0m0.042s\\
Threads = 8: Total real time 0.501662 s
Average real swap time 40.1329 ns
real    0m0.641s
user    0m2.004s
sys     0m0.046s\\
Threads = 40: Total real time 0.468432 s
Average real swap time 187.373 ns
real    0m0.600s
user    0m1.776s
sys     0m0.063s\\

\noindent Virtual, size of state array = 5:\\
Threads = 1: Total real time 1.18059 s Average real swap time 11.8059 ns real    0m1.377s user    0m1.311s sys   0m0.040s\\
Threads = 8: Total real time 0.420923 s Average real swap time 33.6738 ns real  0m0.558s user  0m1.721s sys    0m0.050s\\
Threads = 40: Total real time 0.434194 sAverage real swap time 173.678 ns real 0m0.565suser  0m1.742ssys     0m0.053s\\

\noindent Virtual, size of state array = 100:\\
Threads = 1: Total real time 1.55018 s Average real swap time 15.5018 ns real    0m1.688s user  0m1.686s sys     0m0.040s\\
Threads = 8: Total real time 0.597680 s Average real swap time 47.8144 ns real  0m0.742s user  0m2.099s sys    0m0.051s\\
Threads = 40: Total real time 0.516701 s Average real swap time 206.680 ns real 0m0.654s user  0m1.985s sys   0m0.047s\\

\textbf{7.2 SynchronizedState}\\

\noindent Platform, size of state array = 5:\\
Threads = 1: Total real time 3.36626 s
Average real swap time 33.6626 ns
real    0m3.505s
user    0m3.474s
sys     0m0.049s\\
Threads = 8: Total real time 6.91912 s
Average real swap time 553.529 ns
real    0m7.048s
user    0m9.854s
sys     0m0.238s\\
Threads = 40: Total real time 7.40415 s
Average real swap time 2961.66 ns
real    0m7.566s
user    0m10.274s
sys     0m0.112s\\

\noindent Platform, size of state array = 100:\\
Threads = 1: Total real time 3.38290 s
Average real swap time 33.8290 ns
real    0m3.511s
user    0m3.474s
sys     0m0.039s\\
Threads = 8: Total real time 6.07390 s
Average real swap time 485.912 ns
real    0m6.207s
user    0m7.824s
sys     0m0.178s\\
Threads = 40: Total real time 6.87550 s
Average real swap time 2750.20 ns
real    0m7.006s
user    0m10.504s
sys     0m0.265s\\

\noindent Virtual, size of state array = 5:\\
Threads = 1: Total real time 3.35452 s
Average real swap time 33.5452 ns
real    0m3.487s
user    0m3.479s
sys     0m0.047s\\
Threads = 8: Total real time 13.1645 s
Average real swap time 1053.16 ns
real    0m13.296s
user    0m31.716s
sys     0m4.602s\\
Threads = 40: Total real time 9.03720 s
Average real swap time 3614.88 ns
real    0m9.171s
user    0m18.973s
sys     0m3.293s\\

\noindent Virtual, size of state array = 100:\\
Threads = 1: Total real time 3.35417 s
Average real swap time 33.5417 ns
real    0m3.496s
user    0m3.475s
sys     0m0.044s\\
Threads = 8: Total real time 13.8203 s
Average real swap time 1105.63 ns
real    0m13.981s
user    0m33.371s
sys     0m4.461s\\
Threads = 40: Total real time 10.0014 s
Average real swap time 4000.54 ns
real    0m10.135s
user    0m20.628s
sys     0m3.367s\\

\textbf{7.3 UnsynchronizedState}\\

\noindent Platform, size of state array = 5:\\
Threads = 1: Total real time 1.29734 s
Average real swap time 12.9734 ns
real    0m1.433s
user    0m1.414s
sys     0m0.049s\\
Threads = 8: Total real time 1.99209 s
Average real swap time 159.367 ns
output sum mismatch (56110 != 0)
real    0m2.119s
user    0m7.824s
sys     0m0.052s\\
Threads = 40: Total real time 2.53136 s
Average real swap time 1012.55 ns
output sum mismatch (838 != 0)
real    0m2.658s
user    0m9.899s
sys     0m0.057s\\

\noindent Platform, size of state array = 100:\\
Threads = 1: Total real time 1.45083 s
Average real swap time 14.5083 ns
real    0m1.577s
user    0m1.569s
sys     0m0.055s\\
Threads = 8: Total real time 3.66226 s
Average real swap time 292.981 ns
output sum mismatch (-18021 != 0)
real    0m3.793s
user    0m14.305s
sys     0m0.044s\\
Threads = 40: Total real time 4.10941 s
Average real swap time 1643.76 ns
output sum mismatch (-22037 != 0)
real    0m4.245s
user    0m15.225s
sys     0m0.061s\\

\noindent Virtual, size of state array = 5:\\
Threads = 1: Total real time 1.29288 s
Average real swap time 12.9288 ns
real    0m1.446s
user    0m1.445s
sys     0m0.043s\\
Threads = 8: Total real time 2.58042 s
Average real swap time 206.434 ns
output sum mismatch (-16366 != 0)
real    0m2.713s
user    0m9.954s
sys     0m0.057s\\
Threads = 40: Total real time 2.52382 s
Average real swap time 1009.53 ns
output sum mismatch (-23992 != 0)
real    0m2.687s
user    0m9.805s
sys     0m0.058s\\

\noindent Virtual, size of state array = 100:\\
Threads = 1: Total real time 1.67815 s
Average real swap time 16.7815 ns
real    0m1.813s
user    0m1.806s
sys     0m0.044s\\
Threads = 8: Total real time 3.87638 s
Average real swap time 310.111 ns
output sum mismatch (-19867 != 0)
real    0m4.007s
user    0m15.282s
sys     0m0.057s\\
Threads = 40: Total real time 3.76149 s
Average real swap time 1504.60 ns
output sum mismatch (-21600 != 0)
real    0m3.902s
user    0m14.841s
sys     0m0.052s\\

%%%%%%%%%%%%%%%%%%%%%%%%%%%%%%%%%%%%%%%%%%%%%%%%%%%%%%%%%%%%%%%%%%%%%%%%%%%%%%%%
\end{document}
%%%%%%%%%%%%%%%%%%%%%%%%%%%%%%%%%%%%%%%%%%%%%%%%%%%%%%%%%%%%%%%%%%%%%%%%%%%%%%%%

%%  LocalWords:  endnotes includegraphics fread ptr nobj noindent
%%  LocalWords:  pdflatex acks
